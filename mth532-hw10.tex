\documentclass{article}
\usepackage[margin=1in]{geometry}
\usepackage{amsfonts}
\usepackage{amsmath}
\usepackage{amssymb}
\usepackage{amsthm}
\usepackage{braket}
\usepackage{enumitem}
\usepackage{mathtools}
\usepackage{parskip}

\begin{document}

\title{\vspace{-2cm}MTH 532 Homework 10}
\author{Roy Howie}
\date{April 6, 2017}
\maketitle

\section*{4.2 Exterior Algebra}
  \subsection*{Exercise 1}
    Suppose $T\in\Lambda^p(V^*)$ and $v_1,\cdots,v_p\in V$ are linearly
    dependent, then there are constants $\set{a^1,\cdots,a^p}$, not all zero,
    such that $a^iv_i=0$. Without loss of generality, suppose $v_1=a^2v_2+\cdots
    +a^pv_p$. Therefore,
    \begin{align*}
      T(v_1,\cdots,v_p)
        &= T(a^2v_2+\cdots+a^pv_p,v_2,\cdots,v_p)\\
        &= a^2T(v_2,v_2,\cdots,v_p)+\cdots+a^pT(v_p,v_2,\cdots,v_p)\\
        &= 0+\cdots+0\\
        &= 0\tag*{\qed}
    \end{align*}

  \subsection*{Exercise 2}
    Suppose $\phi_1,\cdots,\phi_p\in V^*$ are linearly dependent, then there are
    constants $a^1,\cdots,a^p$, not all zero, such that $a^i\phi_i=0$. That is,
    without loss of generality, $\phi_1$ can be written as the sum $c^2\phi_2+
    \cdots+c^p\phi_p$. Recall $\phi_i\wedge\phi_i=0$, $c\wedge\phi_i=c\phi_i$
    for constant $c$, and that the wedge product distributes over addition.
    Therefore,
    \begin{align*}
      \phi_1\wedge\cdots\wedge\phi_p
        &= (c^2\phi_2+\cdots+c^p\phi_p)\wedge\phi_2\wedge\cdots\wedge\phi_p\\
        &= c^2\phi_2\wedge\phi_2\wedge\cdots\wedge\phi_p
            +\cdots+
            c^p\phi_p\wedge\phi_2\wedge\cdots\wedge\phi_p\\
        &= 0+\cdots+0\\
        &= 0\tag*{\qed}
    \end{align*}

  \subsection*{Exercise 6}
    \begin{enumerate}[label=\textbf{(\alph*)}]
      \item{
        Let $A\colon V\to V$ be a linear isomorphism which sends the basis
        $B=\set{v_1,\cdots,v_k}$ to $B'=\set{v_1',\cdots,v_k'}$.

        If $B$ and $B'$ are equivalently oriented, then $\det A$ is positive.
        Next, for $T\in\Lambda^k(V)$, note that $A^*T=\det(A)T$. However, $A^*T$
        was also defined to be $T(Av_1,\cdots,Av_k)=T(v_1',\cdots,v_k')$. Thus,
        $T(B)$ and $T(B')$ have the same sign.

        Conversely, suppose $T(B)$ and $T(B')$ have the same sign. Note that
        $T(B')=A^*T=\det(A)T(B)$, implying that $\det A$ is positive. Therefore,
        $B$ and $B'$ are equivalently oriented.
      }
      \item{
        Let $T,S\in\Lambda^k(V^*)$ be two nonzero elements. Suppose $B=\set{v_1,
        \cdots,v_k}$ is a positively oriented ordered basis. Without loss of
        generality, suppose $T(B)$ is positive and $S(B)$ is negative. Let $A
        \colon V\to V$ be a linear isomorphism, then $T(B)$ and $S(AB)$ should
        have the same sign iff $\det A$ is negative. From \textbf{4.2.6a}, this
        is clearly the case, as $S(AB=S(Av_1,\cdots,Av_k)=A^*S(B)=\det(A)S(B)
        $. This produces a natural orientation, as we have defined two
        equivalence classes: positive and negative.
      }
      \item{
        Let $T\in\Lambda^k(V^*)$ be nonzero. Let $B$ and $B'$ be two ordered
        bases. We define an orientation on $V$ by saying $B$ and $B'$ are
        equivalently oriented iff $T(B)$ and $T(B')$ have the same sign. Let
        $A\colon V\to V$ be a linear isomorphism such that $A(B)=B'$.

        Do the opposite of \textbf{4.2.6b}. Note $T(B')=A^*T(B)=\det(A)T(B)$.
        Thus, $T(B')$ and $T(B)$ have the same sign iff $\det A$ is positive,
        implying that $B$ and $B'$ are equivalently oriented.
      }
      \qed
    \end{enumerate}

\end{document}
