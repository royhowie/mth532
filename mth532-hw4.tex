\documentclass{article}
\usepackage[margin=1in]{geometry}
\usepackage{amsfonts}
\usepackage{amsmath}
\usepackage{amssymb}
\usepackage{parskip}
\usepackage{enumitem}
\usepackage{mathtools}

\newcommand{\N}{\mathbb{N}}
\newcommand{\Z}{\mathbb{Z}}
\newcommand{\Q}{\mathbb{Q}}
\newcommand{\R}{\mathbb{R}}
\newcommand{\C}{\mathbb{C}}
\DeclareMathOperator{\GL}{GL}

\begin{document}

\title{MTH 532 Homework 4}
\author{Roy Howie}
\date{February 16, 2017}
\maketitle

\subsection*{Exercise 7}
\subsection*{Exercise 9}
\subsection*{Exercise 12}
\subsection*{Exercise A1}
Note that $M_n\C\cong\C^{n^2}$ and that $\det\colon M_n\C\to\C$ is a smooth
function. Since $\C$ is Hausdorff, $0\in\C$ is closed. Likewise, because of the
continuitity of $\det$, $\det^{-1}(0)$ is closed too. Note that $\GL_n\C=
M_n\C-\det^{-1}(0)$, so $\GL_n\C$ is open and, therefore, a smooth manifold.

Furthermore, $\GL_n\C$ is a group under matrix multiplication with identity
$I_n$. Matrix multiplication is smooth, as it is a polynomial function in the
entries of the product; matrix inversion is smooth because it is a rational
function (Cramer's rule), as the determinant is non-vanishing in $\GL_n\C$.

% Note that $M_n\C$, the set of all $n\times n$ matrices with entries in $\C$, is
% a manifold, as $M_n\C\cong\C^{n^2}$. Therefore, $\GL_n\C$ is a smooth manifold,
% as $\GL_n\C=M_n\C-\det^{-1}(0)$. Furthermore, $\GL_n\C$ has dimension $2n^2$ as
% it has a basis consisting of all matrices with a single element as $1$ or $i$.
%
% We know $\GL_n\C$ is a group (with smooth operations) from algebra.

Hence, $\GL_n\C$ is a Lie group of dimension $2n^2$, as its basis consists of
matrices with one element in $\{1,i\}$.
\hfill $\square$

\subsection*{Exercise A2}
\subsection*{Exercise A3}
\subsection*{Exercise A4}
Let $U_1$ be the unitary group of size 1 and let $S^1$ be the circle group, then
$id\colon U_1\to S^1$ defined by $r\mapsto r$ is a diffeomorphism. Suppose $x\in
U_1$, then $x\bar{x}=1$. Recall, however, that for $z\in\C$, $\|z\|^2=z\bar{z}$.
Therefore, $\|x\|^2 = 1$, which implies $x\in S^1$. The reverse direction is no
different.

Note that $SU_2$ is the set
$
  \left\{
    \left(
      \begin{smallmatrix}
        a & \text{-}\bar{b}\\
        b & \bar{a}
      \end{smallmatrix}
    \right)
    \mid
    a,b\in\C \wedge
    a\bar{a}+b\bar{b}=1
  \right\}
$. Consider the smooth maps $\sigma\colon\R^4\to\C^2$ and $\phi\colon\C^2\to
M_2\C$ defined respectively by $(x,y,z,w)\mapsto(x+iy, z+iw)$ and $(a,b)\mapsto
\left(
  \begin{smallmatrix}
    a & \text{-}\bar{b}\\
    b & \bar{a}
  \end{smallmatrix}
\right)$. Suppose $s=(x,y,z,w)\in S^3$, then $x^2+y^2+z^2+w^2=1$. Let $a=x+iy$
and $b=z+iw$ and consider the identity $z\bar{z}=x^2+y^2$ for complex $z=x+iy$.
Thus, $a\bar{a}+b\bar{b}=1$, so $\phi\circ\sigma(s)\in U_2$. To show the
opposite direction is then trivial. Hence, $S^3$ is diffeomorphic to $SU_2$.
\hfill $\square$

\end{document}
