\documentclass{article}
\usepackage[margin=1in]{geometry}
\usepackage{amsfonts}
\usepackage{amsmath}
\usepackage{amssymb}
\usepackage{amsthm}
\usepackage{braket}
\usepackage{enumitem}
\usepackage{mathtools}
\usepackage{parskip}

\newcommand{\Z}{\mathbb{Z}}
\newcommand{\R}{\mathbb{R}}

\DeclareSymbolFont{matha}{OML}{txmi}{m}{it}% txfonts
\DeclareMathSymbol{\varv}{\mathord}{matha}{118}

\begin{document}

\title{\vspace{-2cm}MTH 532 Homework 11}
\author{Roy Howie}
\date{April 13, 2017}
\maketitle

\section*{4.4 Integration on Manifolds}
  \subsection*{Exercise 1}
    A 0-form $f$ on a manifold $X$ is the same as a real-valued function $f
    \colon X\to\R$; its integral on a single, positively (negatively) oriented
    point is $f(x)$ (respectively $-f(x)$). Hence, the integral of $f$, a
    0-form, over $Z$, a finite collection of points, is the desired sum:
    $\sum_{z\in Z}\sigma(z)f(z)$.
    \qed

  \subsection*{Exercise 3}
    \begin{equation}
      \int_a^bc^*\omega
        =\int_X\omega
        =\int_Xdf
        =\int_{\delta X}f
        =f(q)-f(p)
        \tag*{\qed}
    \end{equation}

  \subsection*{Exercise 4}
    Note that $f$ preserves orientation, as it is smooth, and can be used to
    pull back from $[a,b]$ to $[a_1,b_1]$. Furthermore, $(g\circ f)^*=f^*g^*$.
    Hence,
    \begin{equation*}
      \int_a^bc^*\omega
        =\int_{a_1}^{b_1}f^*c^*\omega
        =\int_{a_1}^{b_1}(c\circ f)^*\omega
        \tag*{\qed}
    \end{equation*}

  \subsection*{Exercise 8}
    Define a 1-form on $\R^2-\set{0}$ by $\omega(x,y)=\left(\frac{-y}{x^2+y^2}
    \right)dx+\left(\frac{x}{x^2+y^2}\right)dy=udx+vdy$.
    \begin{enumerate}[label=\textbf{(\alph*)}]
      \item{
        Let $C$ be a circle of radius $r$ about the origin of the punctured
        plane. Let $\gamma\colon[0,2\pi]\to C$ be an orientation preserving
        parameterization of $C$ minus a single point defined by $t\mapsto(
        r\cos(t),r\sin(t))$.
        \begin{align*}
          \int_C\omega
            &=\int_0^{2\pi}p^*\omega\\
            &=\int_0^{2\pi}(ux'+vy')rdt\\
            &=\int_0^{2\pi}\frac{r^2}{r^2}\left(\frac{\sin^2(t)+\cos^2(t)}
              {\cos^2(t)+\sin^2(t)}\right)rdt\\
            &= 2\pi r
        \end{align*}
      }
      \item{
        Consider $\arctan(y/x)$, then
        \begin{align*}
          \frac{\delta}{\delta x}\arctan\left(\frac{y}{x}\right)
            =\frac{1}{1+\left(\frac{y}{x}\right)^2}\left(\frac{\delta}{\delta x}
            \frac{y}{x}\right)
            =\frac{-\frac{y}{x^2}}{1+\frac{y^2}{x^2}}
            =\frac{-y}{x^2+y^2}\\
          \frac{\delta}{\delta y}\arctan\left(\frac{y}{x}\right)
          =\frac{1}{1+\left(\frac{y}{x}\right)^2}\left(\frac{\delta}{\delta y}
          \frac{y}{x}\right)
          =\frac{\frac{1}{x}}{1+\frac{y^2}{x^2}}
          =\frac{x}{x^2+y^2}
        \end{align*}
      }
      \item{
        Because the punctured plane is not simply connected.
      }
      \qed
    \end{enumerate}

  \subsection*{Exercise 9}
    \textbf{Only if:} Let $h\colon\R\to S^1$ be defined by $t\mapsto(\cos(t),
    \sin(t))$. Restrict $h$ to $[0,2\pi]\subset\R$, then $h$ is a
    parameterization, so we can consider its pullback. Hence, for any 1-form
    $\omega$ on $S^1$, we have $\int_{S^1}\omega=\int_0^{2\pi}h^*\omega$.

    \textbf{If:} Define $g(t)=\int_0^th^*\omega$ and suppose $\int_{S^1}\omega=
    0$, then $g(t)=0$, or that $t\in2\pi\Z$. This follows from the fact that
    $g(t+2\pi)=\int_0^{t+2\pi}h^*\omega=\int_0^th^*\omega+\int_t^{t+2\pi}h^*
    \omega=g(t)$, as $\int_t^{t+2\pi}h^*\omega$ can be reparameterized by some
    orientation-preserving function $r\colon[0,2\pi]\to[t,t+2\pi]$, implying
    $\int_t^{t+2\pi}h^*\omega=\int_0^{2\pi}r^*h^*\omega=0$.

    Therefore, $g=f\circ h$ for some function $f$ on $S^1$.
    \qed

  \subsection*{Exercise 10}
    Let $I_\omega=\int_{S^1}\omega$ and $I_\varv=\int_{S^1}\varv$ and denote
    $c=I_\omega/I_\varv$, which makes sense as $I_\varv\ne0$. This implies
    $\int_{S^1}\omega-c\varv=0$. Exercise \textbf{4.4.9} then implies that
    $\omega$ is the differential of some function $f$, i.e. $\omega=df$.
    \qed

\end{document}
