\documentclass{article}
\usepackage[margin=1in]{geometry}
\usepackage{amsfonts}
\usepackage{amsmath}
\usepackage{amssymb}
\usepackage{amsthm}
\usepackage{parskip}
\usepackage{enumitem}
\usepackage{mathtools}

\newcommand{\N}{\mathbb{N}}
\newcommand{\Z}{\mathbb{Z}}
\newcommand{\Q}{\mathbb{Q}}
\newcommand{\R}{\mathbb{R}}

\begin{document}

\title{MTH 532 Homework 3}
\author{Roy Howie}
\date{February 9, 2017}
\maketitle

\subsection*{Exercise 3}
  Let $f\colon\R\to\R$ be a local diffeomorphism, then $f$ is a local
  homeomorphism and, therefore, an open map. Note that $f$ is continuous. Since
  $\R$ is connected, the image of $f$ must be too. Hence, $f(\R)$ is an open
  interval. Furthermore, $df_x$ is everywhere nonzero, so $f$ is injective.
  Thus, $f$ maps diffeomorphically onto its image.
  \qed

\subsection*{Exercise 4}
  Consider $f\colon\R^2\to\R^2$ defined by $(x,y)\mapsto(e^x\cos y, e^x\sin y)$.
  Then $df$ is
  \begin{equation*}
    \begin{bmatrix}
      e^x\cos{y} & -e^x\sin{y}\\
      e^x\sin{y} &  e^x\cos{y}
    \end{bmatrix}
  \end{equation*}
  So $\text{det }df = e^{2x}\cos^2y+e^{2x}\sin^y=e^{2x}$, which is everywhere
  nonzero. Hence, $f$ is a local diffeomorphism. However, $f$ is clearly not
  invertible, as it has period $2\pi$, so it is not a diffeomorphism onto its
  image.
  \qed

\subsection*{Exercise 6}
  \begin{enumerate}[label=(\alph*)]
    \item{
      Let $f\colon A\to B$ and $g\colon C\to D$. If $f$ and $g$ are immersions,
      then the maps $df_a\colon T_aA\to T_bB$ and $dg_c\colon T_cC\to T_dD$ are
      injective. Note that $T_{(x,y)}(X\times Y)=T_xX\times T_yY$. Hence, $df_a
      \times dg_c\colon T_aA\times T_cC\to T_bB\times T_dD$ equals $d(f\times g)
      _{(a,c)}\colon T_{(a,c)}(A\times C)\to T_{(b,d)}(B\times D)$. Since $df_a$
      and $dg_c$ are injective, so is $d(f\times g)_{(a,c)}$. Thus, $f\times g$
      is an immersion.
    }
    \item{
      Let $f\colon A\to B$ and $g\colon B\to C$ be immersions. Note that $d(g
      \circ f)_x=dg_{f(x)}\circ df_x$. $df_a$ and $dg_{f(a)}$ are injective, so
      $dg_{f(a)}\circ df_a$ is injective for all $a\in A$. Hence, $g\circ f$ is
      an immersion.
    }
    \item{
      Let $Z$ be a submanifold of $X$, let $i\colon Z\xhookrightarrow{} X$ be
      the inclusion map, and let $f\colon X\to Y$ be an immersion. Then $f|_Z
      \colon Z\to Y$ equals $f\circ i$. Since $f$ is an immersion and $i$ is the
      inclusion map, $f|_Z$ is also an immersion.
    }
    \item{
      If $f\colon X\to Y$ is an immersion at $x$ and $y=f(x)$, then, by the
      Local Immersion Theorem, there are local coordinates about $x$ and $y$
      such that $f(x_1,\cdots,x_k)=(x_1,\cdots,x_k,0,\cdots,0)$. However, if
      $\text{dim }X=\text{dim }Y$, then $f(x_1,\cdots,x_k)=(x_1,\cdots,x_k)$, so
      $f$ is indeed a  local diffeomorphism.
    }
    \qed
  \end{enumerate}

\subsection*{Exercise 7}
  \begin{enumerate}[label=(\alph*)]
    \item{
      Let $g\colon\R^1\to S^1$ be defined by $t\mapsto(\cos2\pi t,\sin2\pi t)$,
      then $dg_t = (-2\pi\sin2\pi t,2\pi\cos2\pi t)$. Since $dg_t\ne(0,0)$ for
      all $t$, by the Inverse Function Theorem, $g$ is a local diffeomorphism.
    }
    \item{
      Let $G=g\times g\colon L\to S^1\times S^1$ be defined by $(a,b)\mapsto(
      \cos2\pi a,\sin2\pi a,\cos2\pi b,\sin2\pi b)$, where $L\subset\R^2$ is a
      line of irrational slope. Without loss of generality, suppose $L$ has no
      constant term, as that only serves to ``shift'' the image of $L$ about the
      torus. That is, let $i\in\R-\Q$ and define $L=\{(x,y)\mid y=ix\}$.

      Next, suppose $G(s,is)=G(t,it)$. Then $\cos2\pi s=\cos2\pi t$, so $s-t\in
      \Z$. Similarly, $\cos2\pi is=\cos2\pi it$, so $is-it=i(s-t)\in\Z$. But $i$
      is irrational, so $i(s-t)$ is in $\Z$ iff $s-t=0$. Thus, $G$ is injective.
      \qed
    }
  \end{enumerate}

\subsection*{Exercise 8}
  Let $h\colon\R^1\to\R^2$ be defined by $t\mapsto\frac{1}{2}(e^t+e^{-t},e^t-
  e^{-t})$. To show $h$ is an embedding, we must show that it is an immersion,
  injective, and proper. Note that $dh_t=\frac{1}{2}
  \left[
    \begin{smallmatrix}
      e^t-e^{-t}\\
      e^t+e^{-t}
    \end{smallmatrix}
  \right]$ is injective for all $t$, as $e^{-t}$ is everywhere nonzero. Hence,
  $h$ is an immersion.

  Next, suppose $h$ is not injective. Then there exist $a\ne b$ such that $h(a)=
  h(b)$. But then
  \begin{align*}
    e^a+e^{-b}          &= e^b+e^{-a}           \\
    e^{a+b}(e^a+e^{-b}) &= (e^b+e^{-a})e^{a+b}  \\
    e^a(e^{a+b}+1)      &= (e^{b+a}+1)e^b       \\
    e^a                 &= e^b
  \end{align*}
  Since $x\mapsto e^x$ is injective, $a=b$. A contradiction, so $h$ is
  injective.

  Since $h$ is continuous, the preimage of a closed set is closed. We need only
  show the preimage of a bounded set is itself bounded. Suppose $B\subset\R^2$
  is a bounded set. If $B\cap h(\R)=\varnothing$, we're done, as the empty set
  is trivially bounded. Otherwise, let $B_{x,y}(r)$ be the ball of radius $r$
  centered at $(x,y)$ containing $B$. Let $x_0 = h^{-1}(x, h(x))$, then $h^{-1}
  (B)$ is bounded by the interval $(x_0-r,x_0+r)$.

  Hence, $h$ is proper and thus an embedding.

  To show its image is one nappe of the hyperbola $x^2-y^2=1$, consider
  \begin{align*}
    \frac{1}{4}(e^t+e^{-t})^2-\frac{1}{4}(e^t-e^{-t})^2 &= 1\\
    (e^{2t}+2+e^{-2t})-(e^{2t}-2+e^{-2t})               &= 4\\
    (e^{2t}-e^{2t})+(e^{-2t}-e^{-2t})+4                 &= 4\\
  \end{align*}
  which clearly checks out.
  \qed

\end{document}
